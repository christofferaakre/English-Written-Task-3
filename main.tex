\documentclass{article}
\usepackage[utf8]{inputenc}
\usepackage{thalie}
\title{English Written Task 3}
\author{Christoffer Corfield Aakre}

\newcommand{\pv}{\pauseverse}
\newcommand{\rv}{\resumeverse}

\begin{document}
\maketitle


\section*{\centering Rationale}

For this written task, I will be writing an extension to Macbeth, Act 2, Scene 1, the scene in which Macbeth kills King Duncan. Macbeth is completely under Lady Macbeth's spell, as evident when he confesses that he is killing Duncan 'as an act of love'. Thus, there is a parallell between the witches and Lady Macbeth, as they both control Macbeth. Two major themes of the play that resurface in this written task are guilt and paranoira, as we see the guilt get to Macbeth's head, driving him insane. The task will be linked to part 4 of the course. In writing this task, I will have to show that I understand the major characteristics of Shakespearean English, and that I understand the story. I decided not to write the text in iambic pentametre as I felt such a restrictive style would pose too great an obstacle in writing. The purpose of the text is to further contextualise Macbeth's actions later in the play, where he becomes increasingly paranoid. In the extended scene, Macbeth becomes deranged after killing Duncan. The extended scene is also interesting for Macbeth fans who want further details of Duncan's murder, as it is not usually shown onstage, and is not described in Shakespeare's original play.

\textbf{Word count}: $211$

\pagebreak

\play{Macbeth}
\begin{dramatis}
\character[cmd=macbeth,desc=Thane of Glamis and Thane of Cawdor]{Macbeth}
\character[cmd=duncan,desc=Reigning king of Scotland]{Duncan}
\end{dramatis}

\setcounter{act}{1}
\act{}
\scene{}
[...] I go, and it is done; the bell invites me.
Hear it not, Duncan; for it is a knell
That summons thee to heaven or to hell.

\begin{dida}
Duncan's chambers. Enter MACBETH.
\end{dida}

\duncan
Who's there?

\macbeth
'Tis me, my liege.

\duncan And what brings you hither? We have just supped.

\macbeth I've come to thank you, for these rewards you have bestowed upon me.

\duncan Macbeth, you are a fine man indeed. I thank you for your kindness and hospitality. Assured be thee, there are more to come.

\macbeth What are the tidings at the castle? I've heard times are good.

\duncan Indeed, these are good times. Harvests are plentiful, and crime is little.


\macbeth My liege, I am sorry, but I must confide in you.

\duncan Very well. What troubles you?

\macbeth I've come here with a purpose, my lord.

\duncan Then speak't, if thou wouldst.

\begin{dida}
MACBETH unsheathes his dagger.
\end{dida}

\macbeth You must know it is not of my will. I have come here with the intent to kill. To tear the life out of you, and when the skies turn blue anew, I will be King.

\duncan \did{laughing} You jest Macbeth. I appluad you, for you jest well. Surely you did not come only to jest. What's your mind?

\macbeth Oh, Duncan. Trust me, I wish it were so. I do not jest; I have come for your life.

\duncan Worthy Macbeth, this does not seem like you. I have shown thee the greatest courtesy. Why wouldst thou commit such an act?

\macbeth Thou hast shown me kindness, yes. But I've been promis'd much more: If I kill thee, I shall be King.


\duncan Perhaps it would be so, but thy guilt would surely consume thee. If you do kill thy king, you'll haunt 'till the night grows bright and the day goes dark.

\macbeth My King, I do not want to commit this. But I must, for my fate has been told by witches: I shall be King. Yet further, my Lady stands firm: A real man's not afraid to kill. I've no choice.

\duncan Do not let yourself be controlled by a woman! Macbeth, thou hast proven thyself more than worthy on the battlefield. You need not prove thy manhood.

\macbeth I truly am sorry, you must believe me so. I shiver at the thought that we two should be foes. I am not my own man in this deed.

\duncan If it really is so, and your will is devour'd by your Lady, then hear me so: She is a great evil indeed. You cannot let her command you. She must be stopped at once, for evil thrives in chaos.

\macbeth You do not know how the sun shines, and life thrives when she's around; you do not know how my mind stops when she is around; you do not know how time stops when she smiles. I must do this, for it is an act of love.

\duncan It is an act of cold murder! Macbeth, rally your mind at once! You are not well. Do you not see that she is controlling you?

\macbeth To control or not control, it does not matter, My love for my Lady flows deeper than the oceans, and towers above the mountains.

\duncan I will give you that which your dreams call for. You need only say the word, and it is yours.

\macbeth Enough! Thy life ends now!

\begin{dida}
MACBETH attacks DUNCAN. DUNCAN avoids the attack.
\end{dida}

\duncan Very well, Macbeth. Remember it was you who did this to yourself, not I. Guards! There's an intruder!

\did{Silence}

\macbeth Your guards cannot help you. You must be ended, and it will be so; I'll make sure of it.

\begin{dida}
MACBETH attacks DUNCAN again, and stabs him in the heart. DUNCAN lies down on his bed.
\end{dida}

\duncan Macbeth, you have betrayed your King, and for that you will burn eternally. You will surely bid great darkness cover this kingdom until thy death.

\macbeth I do not care for treachery. Not a soul knows my deed. I will be king, as is my destiny. And my Lady will be Queen.

\duncan Yes, surely you shall be King. But I beg thee, Macbeth, do not let your Lady sit beside you on the throne as Queen. You cannot let her. For your Lady and women like her control. Even now, she dominates you. You must not let her.

\macbeth You have insulted the woman I love; for that you shall suffer.

\begin{dida}
MACBETH stabs DUNCAN repeatedly, in a psychotic manner.
\end{dida}

\duncan \did{dying} I forgive, but I do not forget..

\begin{dida}
MACBETH hears water dripping from the ceiling.
\end{dida}

\macbeth \did{Laughing} Drip, drip, drip. One drip falls, and another one follows. One drip falls, and another one follows. Third one's a charm. What use is loyalty and nobility when I have sweet, savoury love? To kill or not to kill old Duncan, it doesn't matter. He's gone, his soul set free, and I will be King. Think how pleased my Lady will be!

\begin{dida}
DUNCAN bleeds out. Exit MACBETH.
\end{dida}

\textbf{Word count:} $811$

\end{document}
